\documentclass[11pt,twocolumn]{article}
\usepackage[margin=0.8in]{geometry}                % See geometry.pdf to learn the layout options. There are lots.
\geometry{letterpaper}                   % ... or a4paper or a5paper or ... 
%\geometry{landscape}                % Activate for for rotated page geometry
%\usepackage[parfill]{parskip}    % Activate to begin paragraphs with an empty line rather than an indent
\usepackage[breaklinks=true, colorlinks=true, linkcolor=red, urlcolor=blue, citecolor=black]{hyperref}
\urlstyle{rm}
\usepackage{amsmath}
\usepackage{mathptmx}
\usepackage{graphicx}
\usepackage{amssymb}
\usepackage{epstopdf}
\usepackage{color}
\usepackage{sidecap}
\usepackage{authblk}
\usepackage{booktabs}
\usepackage[font=small,labelfont=bf]{caption}
\DeclareGraphicsRule{.tif}{png}{.png}{`convert #1 `dirname #1`/`basename #1 .tif`.png}
\usepackage{enumitem}
\setlist[itemize]{noitemsep}
\setlist[enumerate]{noitemsep}

\def\bfr{\bf\color{red}}
\def\geohub{{\tt geohub}}
\def\Count{count}
\def\ntracts{39}
\def\nprof{8}
\def\nvol{31}
\def\resp{respectively}

\title{\bf
	Results of the 2021 Volunteer Greater Hollywood Homeless Count
	}
\author[*,$\dagger$,$\ddagger$]{Louis Abramson, PhD}%,$\ddagger$
\affil[*]{\it Hollywood 4WRD Homelessness Coalition, 6255 Sunset Blvd, Ste 150, LA, CA 90028}
\affil[$\dagger$]{\it Central Hollywood Neighborhood Council, PO Box 93907, LA, CA 90093}
\affil[$\ddagger$]{\it Carnegie Observatories, 813 Santa Barbara St, Pasadena, CA 91101}
\affil[ ]{\href{mailto:labramson.chnc@gmail.com}{labramson.chnc@gmail.com}}
%\affil[$\dagger$]{\it Princeton University, 4 Ivy Lane, Princeton, NJ 08544}
%\href{mailto:labramson.chnc@gmail.com}{labramson.chnc@gmail.com}}

\date{\today}                                           % Activate to display a given date or no date

\begin{document}
\maketitle

\begin{abstract}

\end{abstract}

\section{Introduction}
\label{sec:intro}

The Los Angeles Homelessness Services Authority (LAHSA) has conducted an annual Point In Time PIT) 
census of the unhoused population of Los Angeles County every year since 20{\bfr XX}. These data are critical to 
essentially all homelessness-related activities in the County and its municipalities. They inform programmatic
funding levels, educate residents, undergird local and state legislative efforts, and shape the day-to-day 
practices of thousands of professional and volunteer service providers. As the official assessment of the 
scope of one of the most pressing humanitarian issues of our time, the LAHSA Count is therefore invaluable.

Disruptions from COVID-19 have only further emphasized the need for such data. As incomes fluctuated,
many of Los Angeles' already sizable number of housing-unstable residents may have been pushed off
couches or out of apartments and onto the street. As such, while the epidemiological considerations of 
conducting an all-volunteer, countywide census are real, the damage from failing to do so is also substantial.

Given the non-uniformity in volunteerism and resources across LAHSA's large area of operations, 
the challenges of COVID were ultimately deemed sufficient to cancel the formal 2021 PIT census of 
unsheltered Angelenos. However, not all communities agreed with this decision, and many had the resources
to execute a robust---if unsponsored---survey on their own. Hollywood is one such community.

Greater Hollywood is an epicenter of LA's homelessness crisis. According to the official 2020 
Count, the Hollywood and East Hollywood Continua of Care (CoC) were home to 2203 unhoused residents,
1714 of whom (78\%) were living unsheltered on the street. This figure corresponds to roughly 5\% of LA's 
homeless population concentrated in an area with only 2.5\% ({\bfr CITE}) of its total population. In some 
regions within those CoCs, fully 1-in-30 residents are unhoused compared to 1-in-100 citywide.

While the above statistics are tragic, Hollywood is also marked by its community of professional
and volunteer service providers, solutions-minded residents and businesspeople, and attuned political 
leaders. Increasingly formal coalitions of the above stakeholders are spreading across the district, dedicated 
to humanely ending the homelessness crisis. All of them rely on the annual PIT count for educational, 
financial, and programmatic purposes. When communicating with the public, the starting point for 
many conversations is simply stating the size of the challenge. When communicating with funders, it is 
similarly critical to convey how many people require services. When designing legislation---especially given 
the shock of COVID-19 and in the face of looming court proceedings---knowing how many unhoused 
Angelenos live where is foundational. For these reasons, organizations and individuals in Hollywood 
decided to with a 2021 Homeless \Count irrespective of other governmental decisions.

This document describes the methodology and findings of that \Count, conducted on the night of 
Thursday, February 25. Below, Section \ref{sec:procedure} describes the volunteer data acquisition, 
analysis, and training protocols. Section \ref{sec:results} present estimates of the unsheltered 
population in the Hollywood, East Hollywood, and Greater Hollywood CoCs. Section \ref{sec:discussion}
contextualizes those findings in terms of previous LAHSA results and describes factors that would
modulate them upwards or downwards. Section \ref{sec:summary} summarizes. Additional information
can be found in the Appendix, including a table of tract-level results in each of the survey's 39 US 
Census tracts.

\section{Methodology}
\label{sec:procedure}

Our \Count\ took place on 25 February 2021 beginning at 7.00 PM. This date and time correspond to 
roughly one month later and four hours earlier than the official event would have occurred. Beyond
those choices, our program adhered as closely as possible to the official LHASA 2020 PIT data 
collection and analysis protocols. Ancillary data from regularly monitored census tracts suggests 
that the date offset is unlikely to substantially erode comparability between this and past datasets, 
though {\bfr we have less purchase on time-of-day effects}.

The \Count\ was based out of The Center at Blessed Sacrament (``The Center''), a major service 
provider in Hollywood. All volunteer teams launched from and returned to this location as they would 
in previous years to a LAHSA community count hub. The major difference was that training was 
performed remotely as a COVID precaution, and volunteer counters never left their vehicles.

\subsection{Data Acquisition}
\label{sec:acquisition}

The \Count\ covered the \ntracts\ US Census tracts constituting the LAHSA-defined Hollywood 
(21 tracts) and East Hollywood (18) Continua of Care (CoC). Our \Count\ did not recognize census 
tract ``splits'' or sub-tracts---e.g., ``1905.10a''---which sets a coarser resolution floor to our results 
compared to past PIT results. Beyond that, the only effect of that choice is to slightly modify of the 
definition of the Hollywood CoC to include all of tract 1905.10 as opposed to only the ``a'' sub-tract. 
This modification has a negligible effect on CoC-level results: since 2016, split 1905.10b has never been 
observed to host more than 7 unsheltered people. The Appendix presents tract-level tallies and raw 
individual/dwelling counts while the main text discusses CoC-level results. Greater Hollywood 
results are available but serve only illustrative purposes as they are not directly comparable to any 
official service geography. Figure \ref{fig:map} shows the \Count\ footprint.

\begin{figure*}
	\centering
	\includegraphics[width=0.9\linewidth]{map}
	\caption{The 2021 volunteer \Count\ covered Greater Hollywood, comprising the 
			officially recognized LAHSA Hollywood and East Hollywood Continua
			of Care. The former stretches from Laurel Canyon Blvd to Western Ave,
			the latter from Western to Hoover Ave. Hollywood is bounded to the north
			and south respectively by Franklin	and Melrose Aves, with East Hollywood
			bounded by Hollywood Blvd and Beverly Ave. Hollywood comprises
			21 census tracts; East Hollywood 18. The grey lines above show census 
			sub-tracts used by LAHSA but ignored in this \Count.}
	\label{fig:map}	
\end{figure*}

All tracts were vetted by professionals from {\it The Center} prior to assignment. Tracts deemed 
especially challenging---due, e.g., to their proximity to freeway onramps and peripheries---were 
reserved for professional counting teams. Vetting produced \nprof\ such tracts, which were surveyed 
by outreach personnel from The Center and Covenant House during daylight hours on 25 
February (circa {\bfr XXX PM}). The remaining \nvol\ tracts were divided among the volunteer 
vehicle-based teams and surveyed beginning at 7.00 PM. Table \ref{tbl:tracts} records which tract
was counted by which kind of team.

We recruited XXX teams of at least two people, YYY of which participated in the \Count\ itself. We 
limited participation to existing ``pods'' of two to three people---typically families---to ensure that the 
COVID status of each participant was controlled and the possibility of transmission minimized. 
Singlet volunteers were also admitted but remained on-site to assist with material distribution, 
collection, and data quality control processes. All participants wore personal protective 
equipment and maintained social distancing when appropriate.

Each vehicle-based volunteer team comprised at least a Driver and a Counter and was assigned two tracts 
to count. Three-person teams also included a Navigator per 2020 LAHSA PIT protocols. In such teams, 
the Navigator held the map and directed the Driver while the Counter tallied unhoused individuals/dwellings 
and the Driver drove. In two-person teams, the Counter doubled as the Navigator. Training emphasized 
techniques aimed at reducing the Counters' cognitive loads and so minimize counting errors. 
These included driving slowly using hazard lights and covering interior streets in a serpentine pattern 
before circling the tract border. Teams were instructed to count both sides of interior streets but only
interior sides of border streets.

Upon arriving at The Center, organizers gave each team a clipboard with:
\begin{itemize}
	\item tract maps;
	\item tally sheets;
	\item a 1-page primer summarizing the training with a contact number for in-field issues.
\end{itemize}
Examples of each of the above documents are included in the Appendix.

The tally sheets were the most important data acquisition tool. These contained separate columns for
each of the nine categories of unhoused individuals or dwellings recognized in the 2020 LAHSA PIT
count: 
\begin{enumerate}
	\item adults (ages $\geq$25);
	\item transition age youths (18--24);
	\item unaccompanied minors;
	\item families (at least one adult with at least one minor); 
	\item cars;
	\item vans;
	\item RVs;
	\item tents;
	\item makeshift structures.
\end{enumerate}
The dwelling classes---Items 5--9---are treated differently than the individual classes in the analysis,
and are hereafter referred to by their acronym ``CVRTM'' when appropriate. 

All teams were deployed to their tracts by {\bfr XXX PM} and returned by {\bfr YYY PM}.

Upon returning, organizers approached each team with a tablet or laptop computer. Counters 
verbally read-off their results for each category as organizers entered them into a google 
form/spreadsheet. The organizer read back the results for confirmation before recovering all 
materials---including hand-written tallies---from the volunteers. Volunteer email addresses 
were also retained for follow-up. 

Once all materials were collected, the organizers convened to cross-check the electronic records
with the physical tally sheets and identify any uncounted areas. {\bfr Follow-up teams were then 
dispatched to count the latter. This was necessary in only XXX instances.} 

{\bfr Given that the number of volunteer teams exceeded the number of tract assignments, a
subset of randomly selected tracts were chosen to be counted by multiple teams. Such duplicate 
measurements are useful for understanding random counting errors and are discussed in Sections 
\ref{sec:dupes} and \ref{sec:discussion}.}

\subsubsection{Volunteer Training}
\label{sec:training}

Teams underwent mandatory, approximately 30 minute Zoom-based training sessions before arriving 
for the \Count. Each participant was also required to watch the official 2020 LAHSA count training video 
and sign participation waivers.

The training covered the motivation for the \Count, an overview of the survey geography (the CoCs), 
team roles, and descriptions of the classes of unhoused individuals/dwellings, including photographic 
examples. Volunteers were instructed to count CVRTM and individuals separately and not to try to 
estimate how many people might live in or be associated with a specific dwelling. This ensured that 
final results could be analyzed as a function of the CVRTM weights, which may change with future 
information (see Section \ref{sec:analysis}).

Volunteers were primed only with min/max estimates of tract-level individual+dwelling counts 
(``0--120'') and the likelihood of encountering unaccompanied minors or families (``very unlikely''). 
Both statements were informed by the 2020 LAHSA PIT results for the Greater Hollywood 
community. No other prior count-based information was established to minimize biases.

A recording of a volunteer training session is available at {\bfr WEBSITE}.

\subsection{Data Analysis}
\label{sec:analysis}

The core component of the raw data was a 9 column by $N_{\rm team}$ row spreadsheet containing the
tract-level tallies for each unhoused individual/dwelling class. The scheme of the analysis is:
\begin{enumerate}
	\item parse and associate tracts with CoCs;
	\item identify tracts counted by multiple teams;
	\item assess tract-level counting errors;
	\item upweight the CVRTM values by estimates of the CVRTM weights.
\end{enumerate}

Our baseline result incorporates the 2020 SPA-4 estimates of the CVRTM weights provided
by LAHSA. These are the best available data, but we recognize that COVID-related activities 
may have significantly changed these quantities. For example, various organizations are know to
have made a concerted effort to distribute tents between last year's PIT count and ours. We 
incorporate all known uncertainties in the weights, but---since they represent potential systematic 
errors---{\bfr analyze the impact of various CVRTM choices in Section \ref{sec:discussion}.}

The resultant $9\times39$ array can then be split and summed to provide CoC-level total counts, 
or breakdowns of unsheltered individual/dwelling classes.

While an estimate of the underlying population, uncertainties in each visual count and weight 
must be accounted for to understand how confident one can be that that estimate corresponds to
the truth. We accomplish this by using Monte Carlo integration to construct the full probability
distribution functions (PDFs) for the number of unsheltered people of each class in each tract.

% {\it These results will
%correspond to the most likely values for the respective quantities in any geography.} However,
%three uncertainties---one small and two large---complicate the interpretation of those sums. 
%We discuss these in Section \ref{sec:discussion}, but account for them as best we can using 
%Monte Carlo techniques to construct the full underlying probability distribution functions (PDFs) 
%for each class in each tract.
%
%All results discussed below derive from 10,000 Monte Carlo realizations of Item (5), above.
%

\subsubsection{Monte Carlo Estimations of Unsheltered Probability Densities}
\label{sec:mc}

Our analysis accounts for two known sources of uncertainty: Poisson counting errors in the visual
tallies and estimated random variances in the CVRTM weights. The former represents how a given
tally might change if performed at a different (but comparable) time or by a different Counter. The 
latter represents how the mean occupancy of CVRTM structures in Hollywood might differ from
the mean occupancy in SPA-4 writ large. 

We model both uncertainties as Gaussian distributions with standard deviations of 
$\sqrt{n}$ and $\sigma$, \resp, where $n$ is the raw tally and $\sigma$ is the standard error on the 
mean quoted by LAHSA. As such, the $i$-th estimate of the true number, $N$, of people 
in the $j$-th unsheltered class in any tract is:
\begin{multline}\label{eq:monte}
	N_{i,j} = \left[n_{j} + \mathcal{G}_{i}(0,\sqrt{n_{j}})\right]\times\max[\mathcal{G}_{i}(w_{j}, \sigma_{j}),1],
\end{multline}
where $\mathcal{G}(\mu,\Sigma)$ is a Gaussian random number with mean $\mu$ and standard deviation 
$\Sigma$, $w$ is the 2020 LAHSA CVRTM weight for the appropriate unsheltered class. If more than
one team counted a given tract, $n$ is replaced by the average of their tallies and the attendant counting
error is divided by $\sqrt{n_{\rm teams}}$.

The final output PDFs are constructed from 10,000 random realizations of Equation \ref{eq:monte}. 
For the individual classes---including families---all weights are 
fixed to unity, such that $(w,\sigma)\equiv(1,0)$ for all trials and uncertainties reflect only 
counting errors.

We place a floor on the CVRTM mean occupancies at 1 person per dwelling; i.e., we assume that the average
person does not own more than one tent. This is not to say no one may own more than one tent, just that
such a statement is never representative. {\bfr Relaxing this assumption does what?} 

\subsection{Duplicate Measurements}
\label{sec:dupes}

\section{Results}
\label{sec:results}

\section{Discussion}
\label{sec:discussion}


\subsubsection{Null Entries}

The minor issue is null entries. As stated in Section \ref{sec:training}, some tracts may be 
relatively free of unhoused people. In these instances, many raw data will read ``0.'' This is
the best estimate of the relevant count in that geography at the time of inspection, but, due to 
Poisson noise, it is consistent with a range of small but non-zero values for the {\it true} count 
one might expect to find at any given time in that area. As such, null entries must be allowed 
to 

The spreadsheet of raw count data was downloaded from the internet before passing it through a number
of programs (written in IDL) to:


\section{Summary}
\label{sec:summary}

\appendix

\section{Example Documents}


\end{document}  