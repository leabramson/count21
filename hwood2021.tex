\documentclass[11pt,twocolumn]{article}
\usepackage[margin=0.8in]{geometry}                % See geometry.pdf to learn the layout options. There are lots.
\geometry{letterpaper}                   % ... or a4paper or a5paper or ... 
%\geometry{landscape}                % Activate for for rotated page geometry
%\usepackage[parfill]{parskip}    % Activate to begin paragraphs with an empty line rather than an indent
\usepackage[breaklinks=true, colorlinks=true, linkcolor=red, urlcolor=blue, citecolor=black]{hyperref}
\urlstyle{rm}
\usepackage{mathptmx}
\usepackage{graphicx}
\usepackage{amssymb}
\usepackage{epstopdf}
\usepackage{color}
\usepackage{sidecap}
\usepackage{authblk}
\usepackage{booktabs}
\usepackage[font=small,labelfont=bf]{caption}
\DeclareGraphicsRule{.tif}{png}{.png}{`convert #1 `dirname #1`/`basename #1 .tif`.png}

\def\bfr{\bf\color{red}}
\def\geohub{{\tt geohub}}
\def\Count{count}
\def\ntracts{39}
\def\nprof{8}
\def\nvol{31}

\title{\bf
	Results of the 2021 Volunteer Greater Hollywood Homeless Count
	}
\author[*,$\dagger$,$\ddagger$]{Louis Abramson, PhD}%,$\ddagger$
\affil[*]{\it Hollywood 4WRD Homelessness Coalition, 6255 Sunset Blvd, Ste 150, LA, CA 90028}
\affil[$\dagger$]{\it Central Hollywood Neighborhood Council, PO Box 93907, LA, CA 90093}
\affil[$\ddagger$]{\it Carnegie Observatories, 813 Santa Barbara St, Pasadena, CA 91101}
\affil[ ]{\href{mailto:labramson.chnc@gmail.com}{labramson.chnc@gmail.com}}
%\affil[$\dagger$]{\it Princeton University, 4 Ivy Lane, Princeton, NJ 08544}
%\href{mailto:labramson.chnc@gmail.com}{labramson.chnc@gmail.com}}

\date{\today}                                           % Activate to display a given date or no date

\begin{document}
\maketitle

\begin{abstract}

\end{abstract}

\section{Introduction}
\label{sec:intro}

The Los Angeles Homelessness Services Authority (LAHSA) has conducted an annual Point In Time PIT) 
census of the unhoused population of Los Angeles County every year since 20{\bfr XX}. These data are critical to 
essentially all homelessness-related activities in the County and its municipalities. They inform programmatic
funding levels, educate residents, undergird local and state legislative efforts, and shape the day-to-day 
practices of thousands of professional and volunteer service providers. As the official assessment of the 
scope of one of the most pressing humanitarian issues of our time, the LAHSA Count is therefore invaluable.

Disruptions from COVID-19 have only further emphasized the need for such data. As incomes fluctuated,
many of Los Angeles' already sizable number of housing-unstable residents may have been pushed off
couches or out of apartments and onto the street. As such, while the epidemiological considerations of 
conducting an all-volunteer, countywide census are real, the damage from failing to do so is also substantial.

Given the non-uniformity in volunteerism and resources across LAHSA's large area of operations, 
the challenges of COVID were ultimately deemed sufficient to cancel the formal 2021 PIT census of 
unsheltered Angelenos. However, not all communities agreed with this decision, and many had the resources
to execute a robust---if unsponsored---survey on their own. Hollywood is one such community.

Greater Hollywood is an epicenter of LA's homelessness crisis. According to the official 2020 
Count, the Hollywood and East Hollywood Continua of Care (CoC) were home to 2203 unhoused residents,
1714 of whom (78\%) were living unsheltered on the street. This figure corresponds to roughly 5\% of LA's 
homeless population concentrated in an area with only 2.5\% ({\bfr CITE}) of its total population. In some 
regions within those CoCs, fully 1-in-30 residents are unhoused compared to 1-in-100 citywide.

While the above statistics are tragic, Hollywood is also marked by its community of professional
and volunteer service providers, solutions-minded residents and businesspeople, and attuned political 
leaders. Increasingly formal coalitions of the above stakeholders are spreading across the district, dedicated 
to humanely ending the homelessness crisis. All of them rely on the annual PIT count for educational, 
financial, and programmatic purposes. When communicating with the public, the starting point for 
many conversations is simply stating the size of the challenge. When communicating with funders, it is 
similarly critical to convey how many people require services. When designing legislation---especially given 
the shock of COVID-19 and in the face of looming court proceedings---knowing how many unhoused 
Angelenos live where is foundational. For these reasons, organizations and individuals in Hollywood 
decided to with a 2021 Homeless \Count irrespective of other governmental decisions.

This document describes the methodology and findings of that \Count, conducted on the night of 
Thursday, February 25. Below, Section \ref{sec:procedure} describes the volunteer data acquisition, 
analysis, and training protocols. Section \ref{sec:results} present estimates of the unsheltered 
population in the Hollywood, East Hollywood, and Greater Hollywood CoCs. Section \ref{sec:discussion}
contextualizes those findings in terms of previous LAHSA results and describes factors that would
modulate them upwards or downwards. Section \ref{sec:summary} summarizes. Additional information
can be found in the Appendix, including a table of tract-level results in each of the survey's 39 US 
Census tracts.

\section{Methodology}
\label{sec:procedure}

Our \Count\ adhered as closely as possible to the official LHASA 2020 PIT data collection and 
analysis protocols. The event took place on 25 February 2021 beginning at 7.00 PM. This date and time
correspond to roughly one month after and four hours earlier than the official event would have occurred. 
Ancillary data in other census tracts suggests that the date offset is unlikely to substantially erode 
comparability between this and past datasets, though {\bfr what can we say about time of day effects}.

The \Count\ was based out of {\it The Center at Blessed Sacrament} (``The Center''), a major service 
provider in Hollywood. All volunteer teams launched from and returned to this location as they would 
in previous years to a LHASA community counting hub. The major difference was that training was 
performed remotely as a COVID precaution.

\subsection{Data Acquisition}
\label{sec:acquisition}

The \Count\ covered the \ntracts\ US Census tracts constituting the LAHSA-defined Hollywood 
and East Hollywood Continua of Care (CoC). As such, our results are integrable at the tract, CoC, 
and Greater Hollywood community levels. The Appendix presents tract-level tallies and raw
individual/dwelling counts while the main text discusses CoC-level results. Greater Hollywood 
results are available but not directly comparable to any official service geography and therefore
serve a descriptive purpose only.

All tracts were vetted by professional outreach personnel from {\it The Center} prior to assignment. 
Any tract deemed especially challenging---e.g., due to their proximity to freeway onramps and 
peripheries---were reserved for professional counting teams. Vetting produced \nprof\ such tracts,
which were surveyed by outreach personnel from {\it The Center} and {\it Covenant Hous} during 
daylight hours on 25 February (circa {\bfr XXX PM}). The remaining \nvol\ tracts were divided 
among the volunteer car-based teams and surveyed beginning at 7.00 PM. Table \ref{tbl:tracts} 
records which teams surveyed each tract.

Using \href{http://www.eventbrite.com}{Eventbrite}, we recruited XXX teams of at least two people, 
YYY of which participated in the \Count\ itself. We limited participation to existing``pods'' of two to 
three people---typically families---to ensure that the COVID status of each participant was controlled 
and the possibility of transmission minimized. Singlet volunteers were also admitted but remained 
on-site to assist with material distribution, collection, and data quality control processes. All participants 
wore proper personal protective equipment and maintained social distancing when appropriate.

Each team comprised at least a Driver and a Counter. Teams of 3 also included a Navigator, again per
2020 LAHSA PIT protocols. In 3-person teams, the Navigator held the map and instructed the Driver 
as to how to stay on-tract. Meanwhile, the Counter visually inspected surroundings and counted unhoused 
individuals/dwellings, and the Driver drove. In 2-person teams, the Counter doubled as the Navigator. 
Training emphasized driving techniques aimed at reducing the cognitive burden of the Counters and 
minimizing tallying errors. These included driving slowly using hazard lights and covering interior streets 
in a serpentine pattern before circling the tract border.

Teams were assigned 2 tracts each. Upon arriving at {\it The Center}, organizers 
gave each team a clipboard with:
\begin{enumerate}
	\item tract maps;
	\item tally sheets;
	\item a 1-page primer summarizing their training with a contact number for issues when deployed.
\end{enumerate}
Examples of each of the above documents are included in the Appendix.

Teams were then deployed to their respective tracts. Departures ran from {\bfr XXX PM} 
to {\bfr YYY PM}. Teams returned to {\it The Center} after their counts between {\bfr ZZZ PM} and 
{\bfr AAA PM}.

In the field, teams were instructed to drive the tract interiors in a serpentine fashion if possible before
making a final pass of their border streets. All teams were instructed only to examine the interior sides
of tract borders. Teams were instructed to drive slowly, using hazard lights if necessary to give counters
adequate time to inspect both sides of the street.

Upon returning, organizers greeted each team with a tablet or laptop computer. Counters verbally 
read-off their tallies to the organizer, who entered the results into a google
form so it could be stored electronically along with the email of the volunteer for any follow-up. The 
organizer read back the results and resolved any interpretation errors before recovering all 
materials---including hand-written tallies---from the volunteers.

After all materials had been collected, the organizers convened to cross-check the electronic records
with the physical tally sheets. At this point, comments were also examined and any uncounted streets
noted. {\bfr Teams were then dispatched to count any uncounted areas, with tallies manually added
to the appropriate tract. This occurred in only XXX instances.} 

{\bfr Given that the number of volunteer teams exceeded the number of available tracts, a
subset of randomly selected tracts were chosen to be counted twice. These duplicate measurments are 
discussed in Sections \ref{sec:dupes} and \ref{sec:discussion}.}

\subsubsection{Volunteer Training}

Teams underwent mandatory Zoom-based training sessions lasting approximately 30 minutes
before arriving at the count. All volunteers
were also required to watch the official 2020 LAHSA count training video, and sign participation waivers.

The training covered the motivation for performing a \Count\ of unsheltered people experiencing
homelessness, an overview of the survey geography (the CoCs), the duties of the various roles,
descriptions of the classes of unhoused individuals---``Adult,'' ``Transition Age Youth,'' 
``Unaccompanied Minors,'' ``Families''---and dwellings---cars, vans, RVs, tents, and makeshift structures
(CVRTM).
Photos of examples of the latter were also included. Volunteers were instructed to
count dwellings and individuals separately and not to try to guess how many people might live
in or be associated with a specific dwelling. This ensured that final results could be analyzed as a 
function of the CVRTM weights, which may change with future information (see Section 
\ref{sec:analysis}).

A recording of a volunteer training session is available at {\bfr WEBSITE}.

\subsection{Data Analysis}
\label{sec:analysis}

The spreadsheet of raw count data was downloaded from the internet before passing it through a number
of programs (written in IDL) to:
\begin{enumerate}
	\item parse and associate tracts with CoCs;
	\item identify tracts counted by multiple teams;
	\item assess tract-level counting errors;
	\item upweight the CVRTM values by the 2020 LAHSA SPA4 CVRTM weights;
	\item construct a Monte Carlo estimate of the underlying probability distributions
		for the number of unhoused people of a given class/in a given dwelling type
		in each census tract.
\end{enumerate}
All results quoted below derive from the 10,000 Monte Carlo samplings of Item (5).

\subsubsection{Monte Carlo Estimations of Unhoused Probability Densities}
\label{sec:mc}

The point-in-time (PIT) data obtained by the Counters represent a draw from the underlying probability
distribution describing how many people are actually experiencing homelessness in a given tract. We have
only the PIT counts, but seek the actual number. We can estimate that as long as we have a model for
the intrinsic uncertainties in the PIT counts and demographic CVRTM weights. We have---or can 
model---both of these uncertainties, and can therefore reconstruct the full probability distribution 
functions (PDFs) using Monte Carlo sampling.

Here, Monte Carlo sampling is simply the process of randomly generating 10,000 alternate versions of
the PIT estimates that reflect what might happen if the count was conducted on a different day or by
a different counter. If you know how a quantity is expected to change from one measurement to another,
independien 

In all cases, baseline Poisson counting uncertainties exist in the raw counts. That is, if $n$ adults were 
Counted by one team in a given tract, the best guess is that the true number of unhoused adults 
in that tract falls between $n\pm\sqrt{n}$ about $2/3$ of the time. We therefore translate each raw 
count into 10,000 simulated counts where each entry refleby a Gaussian random number with standard deviation $\sqrt{n}$ 10,000
to all raw counts to simulate what we could expect to 

In the case of the CVRTM elements, these samplings incorporated both Poisson counting 
errors in the raw counts (random errors) and uncertainties in the CVRTM weights (systematic errors). 
In the case of the individual categories, only the Poison counting errors were propagated.

Explicitly, the $i$th estimate for the number of people in the $j$-th class of unsheltered person in any
tract is:

\begin{equation}\label{eq:monte}
	N_{i,j} = \left[n_{j} + \mathcal{N}(0,\sqrt{n_{j}})\right]\times\mathcal{N}(w_{j}, \sigma_{j})
\end{equation}
where $i$ runs form 1 to 10,1000, $n$ is the raw visual inspection result, $N(\mu,\Sigma)$ is a 
Gaussian random number with mean $\mu$ and standard deviation $\Sigma$, $w$ is the 2020 
LAHSA CVRTM weight for the appropriate class, and $\sigma$ is related to the standard error on that 
weight. For the individual classes---including families---$(w,\sigma)\equiv(1,0)$; i.e., weights were simply 
set to unity for all 10,000 trials.

Equation \ref{eq:monte} entails an assumption that the CVRTM weights, $w$, are normally distributed 
about their quoted values. If we had access to their full probability distribution of the CVRTM weights, we 
would not need to make this assumption. But, given that we did not have access to those distributions, $\sigma$
was calculated by assuming the quoted CVRTM weight was the maximum likelihood value, with a 95\%

The outcome is the equivalent of 10,000 $N_{\rm class}\times N_{\rm tract}$ arrays, each containing a different
point estimate for the size of the $i$th classes' population in the $j$th tract. Summing across trials thus
yields probability distributions that $N$ people of any class are dwelling in any tract.

{\bfr BASELINE BACKGROUND UNCERTAINTIES}

\subsection{Duplicate Measurements}
\label{sec:dupes}

\section{Results}
\label{sec:results}

\section{Discussion}
\label{sec:discussion}

\section{Summary}
\label{sec:summary}

\appendix

\section{Example Documents}


\end{document}  